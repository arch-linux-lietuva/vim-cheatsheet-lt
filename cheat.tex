\documentclass[11pt,a4paper]{article}
\usepackage[left=5mm,top=5mm,right=5mm,bottom=5mm]{geometry}
\usepackage[utf8x]{inputenc}
\usepackage[L7x]{fontenc}
\usepackage[lithuanian]{babel}
\usepackage{multicol}
\usepackage{./style}
\usepackage{fourier}
\begin{document}

\pagestyle{empty}

  \section{Darbas su bylomis}

  \begin{multicols}{2}
    
    \textbf{:e byla} atidaro naują bylą. Galima naudoti tab mygtuką
    greitam bylos vardo surinkimui.
    
    \textbf{:w byla} išsaugo pakeitimus į bylą. Jeigu nenurodyti
    bylos, vim automatiškai išsaugos pakeitimus į atidarytą bylą.

    \textbf{:q} išeiti iš vim redaktoriaus. Jeigu turite
    neišsaugoję pakeitimų - vim atsisakys užsidaryti.

    \textbf{:q!} išeiti iš vim neišsaugant pakeitimų.

    \textbf{:wq} išsaugoti pakeitimus ir išeiti.

    \textbf{:x} kažkas panašaus kaip ir \textsl{:wq}, tačiau
    \textsl{:x} dar patikrina ar buvo padaryti kažkokie pakeitimai
    byloje. Jeigu pakeitimai buvo padaryti - tuomet išsaugo pakeitimus
    ir išeina, jeigu pakeitimai padaryti nebuvo - tiesiog išeina.

  \end{multicols}

  \section{Navigacija byloje}

  \begin{multicols}{2}

    \textbf{j arba rodyklė į viršų} kursorių perkelia viena eilute
    aukščiau.

    \textbf{k arba rodyklė žemyn} viena eilute žemiau

    \textbf{h arba rodyklė į kairę} perkelia kursorių vienu simboliu į
    kairę.

    \textbf{l arba rodyklę į dešinę} perkelia kursorių vienu
    simboliu į dešinę.

    \textbf{e} iki žodžio galo.

    \textbf{E} iki tarpais aptverto žodžio galo.

    \textbf{b} iki žodžio pradžios.

    \textbf{B} iki tarpais aptverto žodžio pradžios.

    \textbf{0} iki eilutės pradžios.

    \textbf{\^{}} iki pirmo žodžio eilutės pradžioje.

    \textbf{\$} iki eilutės pabaigos.

    \textbf{H} iki pirmos eilutės ekrane.

    \textbf{M} iki vidurinės eilutės ekrane.
    
    \textbf{L} iki paskutinės eilutės ekrane.

    \textbf{:n eilutės-skaičius} perkelia iki nurodytos eilutės. 

  \end{multicols}

  \section{Teksto redagavimas}

  \begin{multicols}{2}

    \textbf{i} įterpti prieš kursorių.

    \textbf{I} įterpti eilutės pradžioje.

    \textbf{a} pridėti po kursoriaus.

    \textbf{A} pridėti eilutės pabaigoje.

    \textbf{o} sukurti naują eilutę apačioje ir įterpti.

    \textbf{O} sukurti naują eilutę viršuje ir įterpti.

    \textbf{C} pakeisti likusią eilutės dalį.

    \textbf{r} pakeisti vieną simbolį. Po simbolio pakeitimo,
    grįžtama į command mode.

    \textbf{R} įjungiamas insert mode, tačiau visas tekstas ne
    įterpiamas, o keičiamas.

    \textbf{ESC} išeiti iš bet kokio mode ir grįžti į command mode.

  \end{multicols}

  \section{Teksto trinimas}

  \begin{multicols}{2}

    \textbf{x} ištrinti po kursoriumi esantį simbolį.

    \textbf{X} ištrinti visus simbolius prieš kursorių.

    \textbf{dd} arba \textbf{:d} ištrina visą eilutę.

  \end{multicols}

  \section{Visual mode įjungimas}

  \begin{multicols}{2}

    \textbf{v} pradeda žymėti simbolius. Naudokite paprastus
    navigacinius mygtukus.

    \textbf{V} pradeda žymėti eilutes.

  \end{multicols}

  \section{Teksto blokų redagavimas}

  \begin{multicols}{2}

    \textbf{\~} pakeičia raides iš didžiųjų į mažąsias ir atvirkščiai.

    \textbf{> (V)} pastumti į dešinę.

    \textbf{< (V)} pastumti į kairę.

    \textbf{c (V)} pakeisti pažymėtą tekstą.

    \textbf{y (V)} nukopijuoja tekstą

    \textbf{d (V)} ištrina tekstą.

    \textbf{yy} arba \textsl{:y} arba \textsl{Y} nukopijuoja dabartinę
    eilutę.

    \textbf{p} įklijuoja tekstą.

    \textbf{P} įklijuoja tekstą prieš kursorių.

  \end{multicols}

  \section{Anuliavimas ir perdarymas}

  \begin{multicols}{2}

    \textbf{u} anuliuoti paskutinį veiksmą.

    \textbf{U} anuliuoti visus pakeitimus, kurie buvo atlikti
    eilutėje.

    \textbf{Cltr + r} perdaryti.

  \end{multicols}

  \begin{multicols}{2}

    \section{Paieška}

    \textbf{/žodis} ieškoti žodžio byloje.

    \textbf{n} tęsti paiešką ta pačia linkme.

    \textbf{N} tęsti paiešką priešinga linkme.

  \end{multicols}

\end{document}
